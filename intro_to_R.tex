\documentclass[xcolor=table,dvipsnames]{beamer}

\usepackage{lscape, amsmath, amsfonts, amssymb, setspace, theorem, wrapfig, graphicx, float, multirow, subfig, color, rotating, multicol, datetime, natbib, venndiagram, pstricks, xkeyval, tikz, etoolbox, verbatim, pgf, tikz, pgfplots, mathrsfs, nth}

\usepackage{listings}
\usepackage{xcolor}
\usetikzlibrary{arrows}

\definecolor{codegreen}{rgb}{0,0.6,0}
\definecolor{codegreengray}{rgb}{0,0.4,0}
\definecolor{codegray}{rgb}{0.5,0.5,0.5}
\definecolor{codeblue}{rgb}{0.00,0,0.82}
\definecolor{backcolour}{rgb}{0.95,0.95,0.92}
\definecolor{jeopardy}{rgb}{.24,.47,.914}

\lstdefinestyle{mystyle}{
    backgroundcolor=\color{backcolour},   
    commentstyle=\color{codegreengray},
    numberstyle=\tiny\color{codegray},
    stringstyle=\color{codegreen},
    basicstyle=\ttfamily\footnotesize,
    breakatwhitespace=false,         
    breaklines=true,                 
    captionpos=b,                    
    keepspaces=true,                 
    numbers=left,                    
    numbersep=5pt,                  
    showspaces=false,                
    showstringspaces=false,
    showtabs=false,                  
    tabsize=2
}
 
\lstset{style=mystyle}

\title{1X: Introduction to R}
\subtitle{Essex Summer School in Social Science Data Analysis -- University of Essex}
\date{12 July, 2020}
\author{Lorenzo Crippa}
\institute{University of Essex -- Department of Government}

\usetheme[progressbar=frametitle]{metropolis}
\usecolortheme{seahorse}						% try others: wolverine; crane...

\begin{document}

\begin{frame}[plain]
\begin{center}
\titlepage
\end{center}
\end{frame}

\begin{frame}{Today's session}
Welcome! Today we'll have an introduction to the R programming language. \\ \pause
The goals of today's session are to: \pause
\begin{enumerate}
\item[0.] Introduce what R and RStudio are \pause
\item Learn about R objects and how to create them \pause
\item Learn how to import and manage datasets \pause
\item Learn how to generate basic plots and summary statistics \pause
\item Learn how to run basic statistical analyses
\end{enumerate}
\end{frame}

\begin{frame}{Session structure}
Today's class is structured this way: \pause
\begin{enumerate}
\item A hands-on presentation of the content material (1 hour long) \pause
\item A series of exercises (1 hour long) \pause
\item A review of the exercises run (2 hours long) \pause
\item + An introduction to more advanced packages if time allows it \pause
\end{enumerate}

Break: We can decide together on a break when we are roughly mid-way through the class. \pause

!! \textbf{Always} feel free to interrupt me during the presentation for questions or clarifications by raising your virtual hand. I'll try to check on them as often as I can. \pause Worst case scenario: feel free to interrupt me while I speak.
\end{frame}

\section{0. Introduction to R and RStudio}

\begin{frame}{What is R?}
\begin{itemize}
\item A \textbf{programming language} developed specifically for \textbf{professional} data analysis (based on S). \pause No drag and drop \pause
\item It is a powerful language. It is suited to two types of works: \pause
	\begin{enumerate}
	\item An interactive work; users employ packages and functions that were already written (today's session) \pause
	\item A programming work, more advanced; users write new functions and contribute to the language \pause
	\end{enumerate}
\item R has a rather steep learning curve, but the passage from 1. to 2. comes quicker than expected \pause
\item R is free and open source, and there is an incredible community that uses the language and contributes to it \pause
\item We need to have R installed on our computer to make it run
\end{itemize}
\end{frame}

\begin{frame}{What is RStudio?}
We don't run R directly in it. We run it in RStudio. \pause What is RStudio? \pause
\begin{itemize}
\item It is an integrated development environment (IDE) \pause
\item It automatically flags syntax errors and assigns different colors to different chunks of code \pause
\item The basic version (the one you'll need 99.999\% of the times) is free \pause
\item We still need to have R installed on our computer in order for RStudio to properly run \pause
\end{itemize}

\textbf{Always} run the code in RStudio \textbf{from a script} (.R extension). \pause Don't run it from the console
\end{frame}

\section{1. Topic presentation (in RStudio)}

\section{2. Exercises}

\begin{frame}{Instructions}
\begin{itemize}
\item Now we'll run some exercises directly in RStudio to apply what we've seen before \pause
\item You'll work individually to complete the script \texttt{intro\_to\_R\_ex\_blank.R} that I have provided you with \pause
\item In case you have questions, please raise your virtual hand and I will create a separate breakout room \pause
\item I'll try to keep a list of raised hands in order not to miss anyone \pause
\item If questions are relevant to all, I'll answer them in the classroom \pause
\item \textbf{Remember:} there is more than one way of doing what exercises ask!
\end{itemize}
\end{frame}

\section{3. Review of exercises}

\section{4. Extra topics}

\begin{frame}{Useful R packages}
Some R packages we haven't covered but that might be useful to some of you: \\ \pause

\begin{center}
\resizebox{10cm}{!}{
\begin{tabular}{lll}
\textbf{Type} 		& \textbf{Package name} 	& \textbf{Aim} \\
Data import			& \texttt{foreign} 	& Imports \texttt{.dta} files when \texttt{haven} doesn't \\
					& \texttt{readr}	& Imports files from various formats \\
\hline
Data cleaning 		& \texttt{reshape} 	& Turning long datasets into wide and viceversa\\
					& \texttt{tidyr} 	& Tidy datasets \\
\hline
Plots 				& \texttt{ggplot2} 	& Plots based on Grammar of Graphics \\
					& \texttt{lattice}	& Plots, less versatile than \texttt{ggplot2} \\
\hline
Standard errors		& \texttt{sandwich} + \texttt{lmtest}  & Robust and clustered standard errors + test coefficients \\
					& \texttt{estimatr} & Alternative to \texttt{sandwich} and \texttt{lmtest} \\
\hline
Modelling			& \texttt{plm} 		& Panel data models (within-between estimator) \\
					& \texttt{ggeffects}& Marginal effects from models \\
					& \texttt{AER} 		& Various (Instrumental variable, tobit models) \\ 
					& \texttt{MASS} 	& Various (ordered logit and probit) \\ 
					& \texttt{mlogit} or \texttt{mnlogit}	& Multinomial logit models \\
					& \texttt{lme4} 	& Multilevel models \\
					& \texttt{rdd} 		& Regression discontinuity design \\
					& \texttt{zoo} or \texttt{forecast} & Time series tools \\
\end{tabular}
}
\end{center}
\end{frame}

\begin{frame}{That's all}
\centering
Thank you for the attention! \smallskip

Lorenzo Crippa\\
l.crippa@essex.ac.uk
\end{frame}
\end{document}